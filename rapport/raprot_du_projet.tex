\documentclass[french,a4paper,11pt]{article}

\usepackage[latin1]{inputenc}
\usepackage[T1]{fontenc}
\usepackage[french]{babel}

%\usepackage{layout}
\usepackage[top=2cm,bottom=2cm,left=2cm,right=2cm]{geometry}
%\usepackage{setspace}
%\usepackage{soul}
%\usepackage{ulem}
%\usepackage{eurosym}
%\usepackage{bookman}
%\usepackage{charter}
%\usepackage{newcent}
%\usepackage{lmodern}
%\usepackage{mathpazo}
%\usepackage{mathptmx}
%\usepackage{url}
%\usepackage{verbatim}
%\usepackage{moreverb}
\usepackage{listings}
%\usepackage{fancyhdr}
\usepackage{wrapfig}
%\usepackage{color}
%\usepackage{colortbl}
\usepackage{amsmath}
\usepackage{amssymb}
\usepackage{mathrsfs}
\usepackage{amsthm}
\usepackage{makeidx}
\usepackage{mathtools,bm}
%\usepackage{mathenv}
\usepackage{stmaryrd}
\usepackage{cases}
\usepackage{float}
\usepackage[overload]{empheq}
\usepackage{subcaption}
\usepackage{lscape}
\usepackage{xcolor}
\usepackage{url}
\usepackage{hyperref}

%\newtheorem*{exist_uni}{Th�or�me}
%\newtheorem*{upwind_stab}{Proposition 1}
%\newtheorem*{upwind_ord}{Proposition 2}
%\newtheorem*{reseau}{D�finition}

\title{Calcul de VaR et CVaR par algorithmes stochastiques}
\author{\bsc{Xu Zhiyuan}, \bsc{Yu Zigong}}
\date{\today}

\newcommand{\vertiii}[1]{{\left\vert\kern-0.25ex\left\vert\kern-0.25ex\left\vert #1 
    \right\vert\kern-0.25ex\right\vert\kern-0.25ex\right\vert}}

\begin{document}
\makeatletter
  \begin{titlepage}
  \centering
    \vspace{1cm}
      \includegraphics[scale=0.3]{X_sorb}
      \hfill\\
    \vspace{6cm}
      {\large\textbf{	\\\
       Projet informatique du cours "Probabilit� Num�rique"}}\\
    \vspace{4cm}
       {\LARGE \textbf{\@title}} \\
    \vspace{2em}
        {\large \@author} \\
		\vspace{1em}
        {\large \@date} \\
		\vfill
		\vspace{14em}
		\hfill
				{\large Ann�e universitaire 2019/2020}
  \end{titlepage}
	
\newpage
\section*{Questions}

\begin{itemize}
	\item  
	\item 
	\item
\end{itemize}

\section*{Pense-b�te}

\begin{itemize}
	\item 
	\item 
	\item
\end{itemize}

\section*{Mises � jour}

\begin{itemize}
	\item 
	\item 
	\item
\end{itemize}
	
\newpage
\setcounter{tocdepth}{3}
\tableofcontents

\newpage
\part{Introduction}

\section{Probl�matique}
	


\section{Mod�lisation}



\section{Rappels et notations}

\newpage
\part{Le calcule num�riquement VaR et CVaR et la r�duction de variance}	

\section{VaR et CVaR  pour des niveaux de confiance m�dian}

\subsection{Algorithme fondu sur Rockafellar-Uryasev}


\subsection{Application num�rique}

\subsection{Vitesse de convergence}

\section{R�duction de variance lorsque les niveaux s'�loignent de 1/2}

\subsection{Impl�mentation}

\subsection{Application num�rique}

\subsubsection{Vitesse de convergence}

\newpage
\part{Conclusion}









\newpage
\bibliographystyle{abbrv}
\bibliography{sources}

	\end{document}	